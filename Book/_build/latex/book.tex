%% Generated by Sphinx.
\def\sphinxdocclass{jupyterBook}
\documentclass[letterpaper,10pt,english]{jupyterBook}
\ifdefined\pdfpxdimen
   \let\sphinxpxdimen\pdfpxdimen\else\newdimen\sphinxpxdimen
\fi \sphinxpxdimen=.75bp\relax
%% turn off hyperref patch of \index as sphinx.xdy xindy module takes care of
%% suitable \hyperpage mark-up, working around hyperref-xindy incompatibility
\PassOptionsToPackage{hyperindex=false}{hyperref}
%% memoir class requires extra handling
\makeatletter\@ifclassloaded{memoir}
{\ifdefined\memhyperindexfalse\memhyperindexfalse\fi}{}\makeatother

\PassOptionsToPackage{warn}{textcomp}

\catcode`^^^^00a0\active\protected\def^^^^00a0{\leavevmode\nobreak\ }
\usepackage{cmap}
\usepackage{fontspec}
\defaultfontfeatures[\rmfamily,\sffamily,\ttfamily]{}
\usepackage{amsmath,amssymb,amstext}
\usepackage{polyglossia}
\setmainlanguage{english}



\setmainfont{FreeSerif}[
  Extension      = .otf,
  UprightFont    = *,
  ItalicFont     = *Italic,
  BoldFont       = *Bold,
  BoldItalicFont = *BoldItalic
]
\setsansfont{FreeSans}[
  Extension      = .otf,
  UprightFont    = *,
  ItalicFont     = *Oblique,
  BoldFont       = *Bold,
  BoldItalicFont = *BoldOblique,
]
\setmonofont{FreeMono}[
  Extension      = .otf,
  UprightFont    = *,
  ItalicFont     = *Oblique,
  BoldFont       = *Bold,
  BoldItalicFont = *BoldOblique,
]


\usepackage[Bjarne]{fncychap}
\usepackage[,numfigreset=1,mathnumfig]{sphinx}

\fvset{fontsize=\small}
\usepackage{geometry}


% Include hyperref last.
\usepackage{hyperref}
% Fix anchor placement for figures with captions.
\usepackage{hypcap}% it must be loaded after hyperref.
% Set up styles of URL: it should be placed after hyperref.
\urlstyle{same}


\usepackage{sphinxmessages}



        % Start of preamble defined in sphinx-jupyterbook-latex %
         \usepackage[Latin,Greek]{ucharclasses}
        \usepackage{unicode-math}
        % fixing title of the toc
        \addto\captionsenglish{\renewcommand{\contentsname}{Contents}}
        \hypersetup{
            pdfencoding=auto,
            psdextra
        }
        % End of preamble defined in sphinx-jupyterbook-latex %
        

\title{Physics 111a -- Fluid Mechanics}
\date{Jan 18, 2022}
\release{}
\author{Albion Lawrence}
\newcommand{\sphinxlogo}{\vbox{}}
\renewcommand{\releasename}{}
\makeindex
\begin{document}

\pagestyle{empty}
\sphinxmaketitle
\pagestyle{plain}
\sphinxtableofcontents
\pagestyle{normal}
\phantomsection\label{\detokenize{intro::doc}}


\sphinxAtStartPar
These are the lecture notes for Physics 111a. They are meant to complement
the class textbook {[}\hyperlink{cite.bibliography:id2}{Acheson, 1990}{]}.



\sphinxAtStartPar
I will probably say other things during the lecture so you are
encouraged to take our own notes as well.

\sphinxAtStartPar
Fluid mechanics is a rich and complex subject. I will not cover all of it, and
certainly the book does not. The book and the course are a starting point for
further explanation (as is true for any class you will take!). Furthermore, if
you ar econfused by a point in the book or the lectures, in addition to
talking to me and to your classmates, I encourage you to seek out additional
discussions in the literature.


\chapter{Basic Equations and Concepts}
\label{\detokenize{chapter1/Basic_intro:basic-equations-and-concepts}}\label{\detokenize{chapter1/Basic_intro::doc}}
\sphinxAtStartPar
This week we will introduce some basic concepts and descibe some simple
fluid behaviors. Much of this, particularly the basic equations, will be
described more rigorously down the line.

\sphinxAtStartPar
The natural place to start is with the question “What is a fluid?”. But
before this, unde the assumption that things you think are fluids are
a proper subset of things that are fluids, let us look at some examples
with behavior we will try to explain.
\begin{itemize}
\item {} 
\sphinxAtStartPar
A swimming dog \sphinxhref{https://www.doggypure.com/are-labradors-good-swimmers-coats-waterproof/}{taken from}).

\end{itemize}

\sphinxAtStartPar
\sphinxincludegraphics{{swimming-lab}.jpeg}

\sphinxAtStartPar
(Not in fact my dog, but a dead ringer for him). Here you can see a kind
of wavefront bounding a wake arcing out from and behind the dog,
and choppier water behind him.
\begin{itemize}
\item {} 
\sphinxAtStartPar
Capillary waves emanating from a small disturbance:

\end{itemize}

\sphinxAtStartPar
\sphinxincludegraphics{{capillarywaves}.jpeg}

\sphinxAtStartPar
In this case the surface tension is an important aspect of the dynamics.
Note the wavelengths decrease with distance.
\begin{itemize}
\item {} 
\sphinxAtStartPar
“Gravity waves” emanating from a larger disturbance.

\end{itemize}







\sphinxAtStartPar
As Brad Marston pointed out:






\begin{itemize}
\item {} 
\sphinxAtStartPar
Generation and movement of vortices. Here is a picture (\sphinxhref{https://www.universetoday.com/149514/these-bizarre-cloud-patterns-are-von-karmans-vortices-caused-by-the-air-wrapping-around-tall-islands/}{from} of vortices in atmospheric flow (imprinting on the clouds) past islands in the Cape Verde archipelago:

\end{itemize}

\sphinxAtStartPar
\sphinxincludegraphics{{vonkarmen}.jpeg}

\sphinxAtStartPar
These are all phenomena we will discuss in this course.


\section{What is a Fluid?}
\label{\detokenize{chapter1/whatis_fluid:what-is-a-fluid}}\label{\detokenize{chapter1/whatis_fluid::doc}}

\subsection{1. Characterization}
\label{\detokenize{chapter1/whatis_fluid:characterization}}
\sphinxAtStartPar
I highly recommend the introductory chapter of Batchelor’s book
{[}\hyperlink{cite.bibliography:id3}{Batchelor, 2000}{]} `for an
in\sphinxhyphen{}depth discussion of this issue.


\subsubsection{A \sphinxstyleemphasis{continuous medium} with material at every point.}
\label{\detokenize{chapter1/whatis_fluid:a-continuous-medium-with-material-at-every-point}}
\sphinxAtStartPar
This is of course a mathematical abstraction. A fluid is comprised of
individual molecules and at sufficiently short distances the continuum
approximation breaks down. In practice fluid mechanics is a \sphinxstyleemphasis{coarse\sphinxhyphen{}grained}
description of materials.

\sphinxAtStartPar
The assumption is that if we coarse\sphinxhyphen{}grain over a large enough volume, the
material can be effectively described by a set of quantities which vary
smoothly in space and time, eg:
1. \sphinxstyleemphasis{velocity}: \({\vec v}({\vec x},t)\)
2. \sphinxstyleemphasis{density}: \(\rho({\vec x},t)\) (we will often take this to be constant)
3. \sphinxstyleemphasis{temperature}: \(T({\vec x},t)\)
4. For ocean water, \sphinxstyleemphasis{salinity} (salt content): \(S({\vec x},t)\)
5. For the atmosphere, \sphinxstyleemphasis{specific humidity} \(q = \rho_{vapor}/\rho\).
and so on.

\sphinxAtStartPar
Density, temperature, and so on are themodynamic quantities that we typically
define in equilibrium. A dynamic fluid will not be in equilibrium and in many
interesting examples it is constantly being driven out of equilibrium. As
an example, the solar heating of the atmosphere varies from the equator to
th poles, varies with time over the seasons and over longer times as
the Earth’s roational axis and orbital characteristics change. In this class
we will be assuming that the fluid is in local thermal equilibrium, such
that thermodynamic quantities make sense but vary in space and time.


\subsubsection{Deformation under shear}
\label{\detokenize{chapter1/whatis_fluid:deformation-under-shear}}
\sphinxAtStartPar
Fluids always deform under \sphinxstyleemphasis{shear stress}: essentially, a gradient of force
across some planar surface. There may be friction, corresponding to a
force between two layers sliding on top of each other, but there is no
restoring force.


\subsection{2. Examples}
\label{\detokenize{chapter1/whatis_fluid:examples}}
\sphinxAtStartPar
Note that for our purposes, a fluid could be a liquid, a gas, a plasma, and
so on.  Examples of fluids:
\begin{enumerate}
\sphinxsetlistlabels{\arabic}{enumi}{enumii}{}{.}%
\item {} 
\sphinxAtStartPar
Planetary fluids: oceans, atmospheres, ice sheets, magma

\item {} 
\sphinxAtStartPar
Astrophysics and cosmology: protons and electrons in the
early universe, solar interiors, accretion disks, protoplanetary disks,
near\sphinxhyphen{}horizon geometry of black holes (spacetime as an effective fluid)

\item {} 
\sphinxAtStartPar
Condensed matter physics: superfluids, quantum hall fluids

\item {} 
\sphinxAtStartPar
Biological media (blood, mucous,…)

\item {} 
\sphinxAtStartPar
Plasma in heavy ion collisions

\end{enumerate}

\sphinxAtStartPar
In other words this subject is everywhere: biological physics, soft and
hard condensed matter physics, nuclear physics, astrophysics, cosmology,
earth science, and so on.


\section{Equations for Ideal Fluids}
\label{\detokenize{chapter1/euler:equations-for-ideal-fluids}}\label{\detokenize{chapter1/euler::doc}}
\sphinxAtStartPar
We will derive the equations of fluid dynamics somewhat more systematically
later in the course. Our goal here is to write down some basic equations
and make them plausible. Recommended texts: {[}\hyperlink{cite.bibliography:id4}{Chorin \sphinxstyleemphasis{et al.}, 1990}, \hyperlink{cite.bibliography:id5}{Falkovich, 2018}, \hyperlink{cite.bibliography:id6}{Salmon, 1998}{]}.


\subsection{1. Variables}
\label{\detokenize{chapter1/euler:variables}}

\subsubsection{A. Lagrangian description}
\label{\detokenize{chapter1/euler:a-lagrangian-description}}
\sphinxAtStartPar
In the continuum hypothesis we can consider a fluid as a collecion of
infiniesimal parcels. Each parcel will have a trajectory \({\vec x}(t)\),
and an associated velocity
\begin{equation}\label{equation:chapter1/euler:lag_vel}
\begin{split}{\vec v}({\vec x}(t),t) = \frac{d}{dt} {\vec x}(t)\end{split}
\end{equation}
\sphinxAtStartPar
We then assign each fluid particle a density \(\rho({\vec x}(t), t)\),
temperature, and so on. The description is natural in terms of understanding
a fluid as a collection of parcels which are acted on by external forces
and by each other. It is also natural for certain types of measurements
such as \sphinxhref{http://www.argo.net/}{autonomous floats} in oceanography, and in
understanding the transport of materials (pollutants in the ocean and
atmosphere).


\subsubsection{B. Eulerian description}
\label{\detokenize{chapter1/euler:b-eulerian-description}}
\sphinxAtStartPar
In practice we often cannot or do not follow fluid parcels; rather we take
measurements at points in space and time as determined by our apparatus:
oceanographic moorings, measurements taken from ships or airplanes, and so on.
The description of fluid quantities \({\vec v}({\vec x},t)\), \(\rho({\vec x}, t)\)
at points in space and time is known as the \sphinxstyleemphasis{Eulerian description}.
This description is the dominant one used in fluid dynamics. It is
also the natural framework in which to do numerical experiments.

\sphinxAtStartPar
In this language, we have two interesting sets of integral curves which
are discussed often. The first is the trajectory \({\vec x}(t)\). That is,
given \({\vec v}({\vec x},t)\), and some initial condition \({\vec x}(t_0)\),
the solutions to
\begin{equation}\label{equation:chapter1/euler:trajectory}
\begin{split}frac{d}{dt} {\vec x}(y) = {\vec v}(x,t)\end{split}
\end{equation}
\sphinxAtStartPar
has a unique solution which describes a line in \({\mathbb R}^d\).


\subsection{2. Equations}
\label{\detokenize{chapter1/euler:equations}}\begin{equation}\label{equation:chapter1/euler:newton_second}
\begin{split}\frac{d}{dt} {\vec p} = {\vec F}\end{split}
\end{equation}
\sphinxAtStartPar
where \({\vec p}\) is the momentum of a fluid parcel, and \({\vec F}\) is the
force acting on this parcel. We also assume that the mass of the parcel is
conserved: its density \(\rho\) and volume \(\delta V\) can change with time,
but
\begin{equation}\label{equation:chapter1/euler:mass_cons}
\begin{split}\frac{d}{dt} m({\vec x}(t).t) = \frac{d}{dt} \rho \delta V = 0\end{split}
\end{equation}

\subsubsection{A. Continuity equation}
\label{\detokenize{chapter1/euler:a-continuity-equation}}
\sphinxAtStartPar
Consider a fluid inside a fixed volume \(W\). The total mass inside this volume
is
\begin{equation}\label{equation:chapter1/euler:mass_euler}
\begin{split}M(W,t) = \int_W d^d \rho({\vec x},t)\end{split}
\end{equation}
\sphinxAtStartPar
Because the volume is fixed,
\begin{equation}\label{equation:chapter1/euler:mass_change}
\begin{split}\frac{d}{dt} M = \int_W d^d x \frac{\partial \rho}{\partial t}\end{split}
\end{equation}
\sphinxAtStartPar
On the other hand, the total rate of mass flux into the volume is the
density times the velocity flux through the boundary
surface \({\partial W}\):
\label{equation:chapter1/euler:8f86be87-2320-4301-8c25-a32d0c36f6fb}\begin{align}
	\frac{d}{dt} M & =  - \int_{\partial W} d{\vec A}\cdot \rho {\vec v}\\
	& = \int_W d^d x {\vec \nabla} \cdot (\rho {\vec v})
\end{align}
\sphinxAtStartPar
where we have used the divergence theorem to rewrite the surface integral.
These definitions of \(d_t M\) must be the same for any volume \(W\), so we
finally have the continuity equation:
\begin{equation}\label{equation:chapter1/euler:continuity}
\begin{split}\frac{\partial \rho}{\partial t} + {\vec \nabla}\cdot (\rho {\vec v}) = 0\end{split}
\end{equation}
\sphinxAtStartPar
The focus of this course (not exclusive!) will be on  \sphinxstyleemphasis{incompressible fluids}.
We can define these as fluids for which the density of a parcel does not
change along a trajectory. In this case,
\begin{equation}\label{equation:chapter1/euler:incomp}
\begin{split}\frac{d}{dt} \rho({\vec x}(t), t) = \frac{\partial}{\partial t} 
\rho + {\vec v}\cdot {\vec\nabla} \rho \equiv D_t \rho = 0\end{split}
\end{equation}
\sphinxAtStartPar
Note that here we have defined he \sphinxstyleemphasis{convective derivative}

\sphinxAtStartPar
Combining this with the continuity equation gives us
\begin{equation}\label{equation:chapter1/euler:nodiv}
\begin{split}{\vec \nabla}\cdot {\vec v} = 0\end{split}
\end{equation}
\sphinxAtStartPar
Note that Acheson defines an ideal fluid as having constant
density. This plus the continuity equation implies that the velocity is
dovergence\sphinxhyphen{}free. In other references incompressibility, and not
constant density, is part of the crierion for an ideal fluid. We
should note that there are important examples
of incompressible fluids which have variable density.


\subsubsection{B. Euler’s equation}
\label{\detokenize{chapter1/euler:b-euler-s-equation}}
\sphinxAtStartPar
The next step is to write dynamical equations for the motion of the fluid.
We start with Newton’s laws for a parcel. If the momentum is
\begin{equation*}
\begin{split}{\vec p} = M {\vec v}({\vec x}(t), t) = \rho \delta V {\vec v}({\vec x}(t), t)\end{split}
\end{equation*}
\sphinxAtStartPar
then combining mass conservation with the chain rule, we find:
\begin{equation*}
\begin{split}\frac{d {\vec p}}{dt} = \delta V \rho \left(\frac{\partial}{\partial t}
{\vec v} + {\vec v}\cdot {\vec \nabla} {\vec v}\right) \end{split}
\end{equation*}
\sphinxAtStartPar
This is set equal to the force on the parcel.

\sphinxAtStartPar
To complete the equations, we need an expression for the force
on a fluid parcel.  The force will get contributions from sources
external to the fluid, such as gravity, and from the force due to
neighboring parcels.

\sphinxAtStartPar
If we take the external force \sphinxstyleemphasis{per unit mass} on the particle to
be \({\vec f}\), the total force will \({\vec F} = \rho \delta V {\vec f}\).
An important example is a constant gravitational field acting on
the fluid, for which \({\vec f} = - g {\vec z}\), where \(g\) is the gravitational
acceleration amd \({\vec z}\) the vertical direction.

\sphinxAtStartPar
An \sphinxstyleemphasis{ideal fluid}, in addition to being incompressible, is one in which
the force exerted on a fluid parcel by neighboring parcels takes a specific
form. In particular, there is no shear stress exerted by one parcel on another:
the force is always perpendicular to the surface of the parcel, and takes the
form
\begin{equation*}
\begin{split}{\vec f}_A = - p({\vec x},t) {\hat n}\end{split}
\end{equation*}
\sphinxAtStartPar
where \(p\) is some scalar quantity and \({\vec n}\) is he unit normal pointing
outward from the surface boundary. Noe that by Newton’s thurd law, the
parcel will exert the force \(p({\vec x},t) {\hat n}\) on the neighboring
parcel, for which \(-{\hat n}\) is the unit normal pointing outwards from
that parcel. Integating \({\vec f}_A\) over the surface of the parcel,
we find the total force on a parcel in volume \(W\) is
\begin{equation*}
\begin{split}{\vec F} = - \int_{{\partial W}} d{\vec A} p = - \int_W
{\vec \nabla} p \sim - \delta V {\vec \nabla} p\end{split}
\end{equation*}
\sphinxAtStartPar
where \(\delta V\) is the volume of \(W\). We identify \(p\) with the \sphinxstyleemphasis{pressure}.
Adding all of the forces together and dividing the whole momentum
equation by \(\rho\) we get \sphinxstyleemphasis{Euler’s equation}:
\begin{equation*}
\begin{split}D_t {\vec v} = - \frac{1}{\rho} {\vec \nabla} p 
- \frac{1}{\rho} {\vec F}_{ext}\end{split}
\end{equation*}

\section{Hydrostatics}
\label{\detokenize{chapter1/Hydrostatics:hydrostatics}}\label{\detokenize{chapter1/Hydrostatics::doc}}
\sphinxAtStartPar
The focus of his course is fluid \sphinxstyleemphasis{dynamics} in which fluid parcels
have interesting nonrivial motion. Nonethelss I would like to pause here
and make a couple of statements about static fluid configurations
in the presence of nontrivial forces.


\subsection{1. Conservative force field}
\label{\detokenize{chapter1/Hydrostatics:conservative-force-field}}
\sphinxAtStartPar
First, if the force per unit mass is conservative, then
\begin{equation*}
\begin{split}
	\frac{1}{\rho} {\vec F}_{ext} = {\vec \nabla} \Psi
\end{split}
\end{equation*}
\sphinxAtStartPar
Note this is not necessarily the same as \({\vec F}_{ext}\) being a gradient.
Note also that it does include cases such as a constant vertical gravitational
field for which \(\Psi = - g z\). In this case, if the fluid parcels are static,
\begin{equation*}
\begin{split}
	{\vec \nabla} p = \rho 	{\vec \nabla} \Psi
\end{split}
\end{equation*}
\sphinxAtStartPar
For constant \(\rho\) we would just have \(p = \rho \Psi\). More generally, taking
the curl of both sides and recalling that the curl of the gradient vanishes,
\begin{equation*}
\begin{split}
	{\vec \nabla} \rho \times {\vec \nabla \Psi} = 0
\end{split}
\end{equation*}
\sphinxAtStartPar
which is only possible if the gradients of \(\rho\) and \(\Psi\) are parallel, and
thus their level lines coincide.


\subsection{2. Example: Isothermal atmospere}
\label{\detokenize{chapter1/Hydrostatics:example-isothermal-atmospere}}
\sphinxAtStartPar
Let us consider the case that the atmosphere is uniform
in the directions horizontal to the ground, so that \(p,\rho\) depend
on the vertical coordinate \(z\) only, and
\begin{equation*}
\begin{split}
	\partial_z p = - \rho g
\end{split}
\end{equation*}
\sphinxAtStartPar
If \(\rho\) is constant, then \(p(z) = p(0) - g \rho z\). If \(\rho\) is not
constant but the atmosphere is an ideal gas at fixed temperature,
\(p = \rho T/m\) is the ideal gas law, where \(m\) is the mass per molecule
(for a monomolecular gas) so hat \(\rho/m\) is he number of molecules per
unit volume. For constant \(T\) (“isothermal gas”), the hydrostatic equation
becomes:
\begin{equation*}
\begin{split}
	\partial_z p = \frac{T}{m} \partial_z \rho = - \rho g
\end{split}
\end{equation*}
\sphinxAtStartPar
This has the exponential solution
\$\(
	\rho(z) = \rho(0) e^{-m g z/T} \Rightarrow p(z) = \frac{T \rho(0)}{m} 
	e^{-mgz/T}
\)\(
where \)z = 0\( is the ground and \)z\$ increases aboveground.

\sphinxAtStartPar
The actual atmosphere is more complex: the temperature itself drops in the
troposhere up to the \sphinxstyleemphasis{tropopause}, around \(\sim 10\)km
high, and then after remaining constant for another \(\sim 30\) km,
begins to rise wih height in the stratosphere. The actual profile
\(p(z)\) is somewhere between exponentially decaying and linearly decaying.
In general, the atmosphere and ocean are \sphinxstyleemphasis{stratified fluids} with
vertically varying densities.


\section{Bernoulli’s Equation}
\label{\detokenize{chapter1/Bernoulli:bernoulli-s-equation}}\label{\detokenize{chapter1/Bernoulli::doc}}

\subsection{1. Isentropic motion}
\label{\detokenize{chapter1/Bernoulli:isentropic-motion}}
\sphinxAtStartPar
Here we will make recourse to a bit of thermodynamics. We can define
the \sphinxstyleemphasis{enthalpy} as
\begin{equation}\label{equation:chapter1/Bernoulli:enthalpy}
\begin{split}W = E + p V\end{split}
\end{equation}
\sphinxAtStartPar
where \(E\) is the inernal energy. Using the first law of thermodynamics,
\(dE = T dS - p d V\), where \(S\) is the entropy, we find
\begin{equation}\label{equation:chapter1/Bernoulli:denth}
\begin{split}dW = T dS + V dp = T dS + \frac{1}{\rho} dp\end{split}
\end{equation}
\sphinxAtStartPar
For “isentropic” fluids, with no input of hea and no heat exchange between
fluid parcels, \(dS = 0\) for each parcel, and \(\frac{dp}{d\rho} = dW\). Thus
\begin{equation*}
\begin{split}\frac{\partial}{\partial t} {\vec v} + {\vec v} \cdot {\vec \nabla}
{\vec v} = - {\vec \nabla} W\end{split}
\end{equation*}

\subsection{2. Bernoulli’s theorem.}
\label{\detokenize{chapter1/Bernoulli:bernoulli-s-theorem}}
\sphinxAtStartPar
Let us consider a velocity field \({\vec v}\) which is constant in time.
In this case, the trajectories of fluid parcels follow the streamlines
of the velocity field. In this case, \sphinxstyleemphasis{Bernoulli’s theorem} states that
\(W + \frac{1}{2} {\vec v}^2\) is constant along streamlines.

\sphinxAtStartPar
The proof starts with the vector identity
\begin{equation*}
\begin{split}
	{\vec A} \times ({\vec \nabla}\times {\vec B}) = 
	A^i {\vec \nabla} B_i - {\vec A} \cdot {\vec \nabla} {\vec B}
\end{split}
\end{equation*}
\sphinxAtStartPar
(where we have used Einstein summation notation). Applying this to
\({\vec A} = {\vec B} = {\vec v}\).
\begin{equation*}
\begin{split}
	{\vec v} \times ({\vec \nabla} \times {\vec v}) = 
	\frac{1}{2} {\vec \nabla} {\vec v}^2 - {\vec v} \cdot{\vec \nabla}
	{\vec v}
\end{split}
\end{equation*}
\sphinxAtStartPar
Applying this to the Euler equation, and taking the case
\(\partial_t {\vec v} = 0\), we find
\begin{equation*}
\begin{split}
{\vec \nabla} (W + \frac{1}{2} {\vec v}^2) = {\vec v} \times({\vec \nabla}\times{\vec v})
\end{split}
\end{equation*}
\sphinxAtStartPar
Now the RHS is perpendicular to \({\vec v}\) since is a cross product with this
vector, and thus
\begin{equation*}
\begin{split}
	{\vec v} \cdot {\vec \nabla}  (W + \frac{1}{2} {\vec v}^2) = 0
\end{split}
\end{equation*}
\sphinxAtStartPar
But \({\vec v} \cdot {\vec \nabla}\) corresponds precisely to the derivative
along streamlines.


\chapter{Bibliography}
\label{\detokenize{bibliography:bibliography}}\label{\detokenize{bibliography::doc}}
\sphinxAtStartPar


\begin{sphinxthebibliography}{Ach90}
\bibitem[Ach90]{bibliography:id2}
\sphinxAtStartPar
DJ Acheson. \sphinxstyleemphasis{Elementary fluid dynamics: Oxford University Press}. Oxford, England, 1990.
\bibitem[Bat00]{bibliography:id3}
\sphinxAtStartPar
G.K. Batchelor. \sphinxstyleemphasis{An Introduction to Fluid Dynamics}. Cambridge Mathematical Library. Cambridge University Press, 2000. ISBN 9780521663960.
\bibitem[CMM90]{bibliography:id4}
\sphinxAtStartPar
Alexandre Joel Chorin, Jerrold E Marsden, and Jerrold E Marsden. \sphinxstyleemphasis{A mathematical introduction to fluid mechanics}. Springer, 3rd edition, 1990.
\bibitem[Fal18]{bibliography:id5}
\sphinxAtStartPar
Gregory Falkovich. \sphinxstyleemphasis{Fluid Mechanics}. Cambridge University Press, 2 edition, 2018. \sphinxhref{https://doi.org/10.1017/9781316416600}{doi:10.1017/9781316416600}.
\bibitem[Sal98]{bibliography:id6}
\sphinxAtStartPar
Rick Salmon. \sphinxstyleemphasis{Lectures on geophysical fluid dynamics}. Oxford University Press, 1998.
\end{sphinxthebibliography}







\renewcommand{\indexname}{Index}
\printindex
\end{document}